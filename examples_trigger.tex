\subsection{Estimation of trigger rates at the HL-LHC}
\label{sec:examples_trigger_rate}

We choose the task of estimating trigger rates for the upcoming high-luminosity data-taking period of the LHC as second example to illustrate the stitching procedure.
The ``rate'' of a trigger corresponds to the number of $\Pp\Pp$ collision events that satisfy the trigger condition per unit of time.
The estimation of trigger rates constitutes an important task for demonstrating the physics potential of the HL-LHC.
The HL-LHC physics program demands a large amount of integrated luminosity to be delivered by the LHC, 
in order to facilitate measurements of rare signal processes
(such as the precise measurement of $\PHiggs$ boson couplings and the study of $\PHiggs$ boson pair production) by the ATLAS and CMS experiments.
In order to satisfy this demand, the HL-LHC is expected to operate at an instantaneous luminosity of $5$-$7.5 \times 10^{34}$~cm$^{-2}$~s$^{-1}$
at a center-of-mass energy of $\sqrt{s} = 14$~\TeV~\cite{TDR_Phase2_LHC}.
The challenge of developing triggers for the HL-LHC is to design the triggers such that rare signal processes pass the triggers with a high efficiency,
while the rate of background processes gets reduced by many orders of magnitude, in order not to exceed bandwidth limitations on the detector read-out 
and on the rate with which events can be written to permanent storage.

The inelastic $\Pp\Pp$ scattering cross section at $\sqrt{s} = 14$~\TeV amounts to $\approx 80$~mb,
resulting in up to $200$ simultaneous $\Pp\Pp$ interactions per crossing of the proton beams at the nominal HL-LHC instantaneous luminosity~\cite{TDR_Phase2_LHC}.
The vast majority of these interactions are inelastic $\Pp\Pp$ scatterings with low momentum exchange,
which predominantly arise from the exchange of gluons between the colliding protons.
We refer to inelastic $\Pp\Pp$ scattering interactions with no further selection applied as ``minimum bias'' events.
In order to estimate the rates of triggers at the HL-LHC,
MC samples of minimum bias events are produced at LO in pQCD using the program \PYTHIA.
The minimum bias samples are complemented by samples of inelastic $\Pp\Pp$ scattering interactions
in which a significant amount of transverse momentum, denoted by the symbol $\pThat$, is exchanged between the scattered protons.
The stitching of the minimum bias samples with samples generated for different ranges in $\pThat$ allows to estimate the trigger rates with lower statistical uncertainties.

The production of MC samples used for estimating trigger rates at the HL-LHC
proceeds by first simulating one ``hard-scatter'' (HS) interaction within a given range in $\pThat$
and then adding a number of additional inelastic $\Pp\Pp$ scattering interactions of the minimum bias kind to the same event,
in order to simulate the pileup (PU).
We use the symbol $N_{\pileup}$ to denote the total number of PU interactions 
that occur in the same crossing of the proton beams as the HS interaction.
No selection on $\pThat$ is applied when simulating the PU interactions.
We remark that the distinction between the HS interaction and the PU interactions is artificial and solely made for the purpose of MC production.
The HS interaction and the PU interactions will be indistinguishable in the data that will be recorded at the HL-LHC:
The scattering in which the transverse momentum exchange between the protons amounts to $\pThat$ may occur in any of the $N_{\pileup} + 1$ simultaneous $\Pp\Pp$ interactions.
Our formalism treats the HS interaction and the PU interactions on an equal footing.

The ``inclusive'' sample in this example are events containing $N_{\pileup} + 1$ minimum bias interactions,
where for each event the number of PU interactions, $N_{\pileup}$, is sampled at random from the Poisson probability distribution:
\begin{equation}
\Poisson(N_{\pileup} \vert \Nbar) = \frac{\Nbar^{N_{\pileup}} \times e^{-\Nbar}}{N_{\pileup}!}
\label{eq:Poisson}
\end{equation}
with a mean $\Nbar = 200$.
The exclusive samples contain one HS interaction of transverse momentum within a specified range in $\pThat$ in addition to $N_{\pileup}$ minimum bias interactions.
The latter represent the PU.
The number $N_{\pileup}$ of PU interactions is again sampled at random from a Poisson distribution with a mean of $\Nbar = 200$.

We enumerate the ranges in $\pThat$ by the index $i$ and denote the number of $\pThat$ ranges used to produce the exclusive samples by the symbol $m$.
We further introduce the symbol $n_{i}$ to refer to the number of inelastic $\Pp\Pp$ scattering interactions that fall into the $i$-th interval in $\pThat$.
The inelastic $\Pp\Pp$ scatterings may occur either in the HS interaction or in any of the $N_{\pileup}$ PU interactions.
The ``phase space'' corresponding to a given event is represented by a vector $I=n_{1},\dots,n_{m}$ of dimension $m$.
The $i$-th component of this vector indicates the number of inelastic $\Pp\Pp$ scattering interactions that fall into the $i$-th interval in $\pThat$.

The probability $P^{I}$ for an event in the inclusive sample that contains $N_{\pileup}$ pileup interactions
to feature $n_{1}$ inelastic $\Pp\Pp$ scatterings that fall into the first interval in $\pThat$, $n_{2}$ that fall into the second,$\dots$, and $n_{m}$ that fall into the $m$-th 
follows a multinomial distribution~\cite{evans2011statistical} and is given by:
\begin{equation}
P_{\incl}^{I} = \frac{(N_{\pileup} + 1)!}{n_{1}! \cdot \dots \cdot n_{m}!} \cdot p_{1}^{n_{1}} \cdot \dots \cdot p_{m}^{n_{m}} \, ,
\label{eq:P_inclusive}
\end{equation}
where the symbols $p_{i}$ correspond to the probability for a single inelastic $\Pp\Pp$ scattering interaction to feature a transverse momentum exchange that falls into the $i$-th interval in $\pThat$.
The $n_{i}$ satisfy the condition $\sum_{i=1}^{m} \, n_{i} = N_{\pileup} + 1$.

The corresponding probability $P_{j}^{I}$ for an event in the $j$-th exclusive sample that contains $N_{\pileup}$ pileup interactions is given by:
\begin{equation}
P_{j}^{I} = \begin{cases}
\frac{N_{\pileup}!}{n_{1}! \cdot \dots \cdot (n_{j} - 1)! \cdot \dots \cdot n_{m}!} \cdot p_{1}^{n_{1}} \cdot \dots \cdot p_{j}^{(n_{j} - 1)} \cdot \dots \cdot p_{m}^{n_{m}} \, ,
  & \text{if $n_{j} \geq 1$} \\
0 \, , & \text{otherwise} \, .
\end{cases}
\label{eq:P_exclusive}
\end{equation}
The $n_{i}$ again satisfy the condition $N_{\pileup} + 1 = \sum_{i=1}^{k} \, n_{i}$.
The fact that for all events in the $j$-th exclusive sample the transverse momentum $\pThat$ that is exchanged in the HS interaction falls into the $j$-th interval in $\pThat$
implies that $N_{\pileup} + 1$ needs to be replaced by $N_{\pileup}$ and $n_{j}$ by $n_{j} - 1$ in Eq.~(\ref{eq:P_exclusive}) compared to Eq.~(\ref{eq:P_inclusive}),
as one of the inelastic $\Pp\Pp$ scatterings that fall into the $j$-th interval in $\pThat$ is ``fixed'' and thus not subject to the random fluctuations, which are modeled by the multinomial distribution.
The ratio of Eq.~(\ref{eq:P_exclusive}) to Eq.~(\ref{eq:P_inclusive}) is given by the expression:
\begin{equation}
\frac{P_{j}^{I}}{P_{\incl}^{I}} = \frac{n_{j}}{(N_{\pileup} + 1) \cdot p_{j}} \, .
\label{eq:P_ratio}
\end{equation}
The validity of Eq.~(\ref{eq:P_ratio}) includes the case $n_{j} = 0$.

The expression for the stitching weight $s^{I}$ is given by an expression similar to Eq.~(\ref{eq:weight_incl}),
the main difference being that the index $i$ is replaced by the vector $I$,
the probabilities $P_{\incl}^{i}$ and $P_{j}^{i}$ are replaced by the probabilities $P_{\incl}^{I}$ and $P_{j}^{I}$
and the product of luminosity times cross section, $L \cdot \sigma_{\incl}$, is replaced by the frequency $F$ of $\Pp\Pp$ collisions:
\begin{equation}
s^{I} = \frac{F}{\sum_{k=1}^{N_{\incl}} \, w_{\incl}^{k}} \cdot \frac{P^{I} \cdot \sum_{k=1}^{N_{\incl}} \, w_{\incl}^{k}}{P_{\incl}^{I} \cdot \sum_{k=1}^{N_{\incl}} \, w_{\incl}^{k} + \sum_{j} \, P_{j}^{I} \cdot \sum_{k=1}^{N_{j}} \, w_{j}^{k}} \, .
\label{eq:weight_tmp}
\end{equation}
The probabilities $P_{\incl}^{I}$ and $P_{j}^{I}$ are given by Eqs.(~\ref{eq:P_inclusive}) and~(\ref{eq:P_exclusive}).
Dividing both numerator and denominator on the right-hand side of Eq.~(\ref{eq:weight_tmp}) by $P_{\incl}^{I}$ and replacing the ratio $P_{j}^{I}/P_{\incl}^{I}$ by Eq.~(\ref{eq:P_ratio}) yields:
\begin{equation}
s^{I} = \frac{F}{\sum_{k=1}^{N_{\incl}} \, w_{\incl}^{k} + \sum_{j} \, \frac{n_{j}}{(N_{\pileup} + 1) \cdot p_{j}} \cdot \sum_{k=1}^{N_{j}} \, w_{j}^{k}} \, .
\label{eq:weight_trigger_rate}
\end{equation}
At the HL-LHC, the $\Pp\Pp$ collision frequency $F$ amounts to $28$~MHz~\footnote{
  The beams cross every $25$~ns, but $\Pp\Pp$ collisions occur only in $\approx 70\%$ of those beam crossings~\cite{TDR_Phase2_LHC}.}.
Eq.~(\ref{eq:weight_trigger_rate}) represents the equivalent of Eq.~(\ref{eq:weight_incl}),
tailored to the case of estimating trigger rates instead of estimating event yields.
The weights $w_{\incl}^{k}$ and $w_{j}^{k}$ are equal to one for all events in this example,
which allows to simplify Eq.~(\ref{eq:weight_trigger_rate}).
Using the relations $\sum_{k=1}^{N_{\incl}} \, w_{\incl}^{k} = N_{\incl}$ and $\sum_{k=1}^{N_{j}} \, w_{j}^{k} = N_{j}$,
we obtain the expression:
\begin{equation}
s^{I} = \frac{F}{N_{\incl} + \sum_{j} \, N_{j} \cdot \frac{n_{j}}{(N_{\pileup} + 1) \times p_{j}}} \, .
\label{eq:weight_trigger_rate_simplified}
\end{equation}

The ranges in $\pThat$ used to produce the exclusive samples and the number of events contained in each sample
are given in Table~\ref{tab:samples_trigger_rate}.
The association of the index $i$ to the different ranges in $\pThat$ and the 
corresponding values of the probabilities $p_{i}$ are given in Table~\ref{tab:p_trigger_rate}.
The probabilities $p_{i}$ are computed by taking the ratio of cross sections computed by the program \PYTHIA
for the case of single inelastic $\Pp\Pp$ scattering interactions with a transverse momentum exchange that is within the $i$-th interval in $\pThat$
and for the case that no condition is imposed on $\pThat$.

\begin{table}[h!]
\begin{center}
\begin{tabular}{l|c}
Sample                    & Number of events \\
\hline
Inclusive                 & $8 \times 10^{5}$ \\
$ 30 < \pThat <  50$~\GeV & $4 \times 10^{5}$ \\
$ 50 < \pThat <  80$~\GeV & $2 \times 10^{5}$ \\
$ 80 < \pThat < 120$~\GeV & $1 \times 10^{5}$ \\
$120 < \pThat < 170$~\GeV & $5 \times 10^{4}$ \\
$170 < \pThat < 300$~\GeV & $5 \times 10^{4}$ \\
$300 < \pThat < 470$~\GeV & $5 \times 10^{4}$ \\
$470 < \pThat < 600$~\GeV & $5 \times 10^{4}$ \\
$\pThat > 600$~\GeV       & $5 \times 10^{4}$ \\
\end{tabular}
\end{center}
\caption{
  Number of events in the inclusive and exclusive samples used to estimate trigger rates at the HL-LHC.
}
\label{tab:samples_trigger_rate}
\end{table}

\begin{table}[h!]
\begin{center}
\small
\begin{minipage}{16cm}
\begin{tabular}{l|cccccc}
Range in $\pThat$ [\GeV] & $< 30$ & $30$-$50$ & $50$-$80$ & $80$-$120$ & $120$-$170$ & $170$-$300$ \\
Index $i$           & $0$ & $1$ & $2$ & $3$ & $4$ & $5$ \\
\hline
Probability $p_{i}$ & $0.998$ & $1.51 \times 10^{-3}$ & $2.25 \times 10^{-4}$ & $3.38 \times 10^{-5}$ & $6.00 \times 10^{-6}$ & $1.55 \times 10^{-6}$ \\
\end{tabular}

\vspace{2mm}

\begin{tabular}{l|ccc}
Range in $\pThat$ [\GeV] & $300$-$470$ & $470$-$600$ & $> 600$ \\
Index $i$           & $6$ & $7$ & $8$ \\
\hline
Probability $p_{i}$ & $1.05 \times 10^{-7}$ & $8.73 \times 10^{-9}$ & $3.12 \times 10^{-9}$ \\
\end{tabular}
\end{minipage}
\end{center}
\caption{
  Probabilities $p_{i}$ for a single inelastic $\Pp\Pp$ scattering interaction to feature a transverse momentum exchange 
  between the protons that is within the $i$-th interval in $\pThat$.
}
\label{tab:p_trigger_rate}
\end{table}

We cannot give numerical values of the stitching weights $s^{I}$ for this example,
as $I$ is a high-dimensional vector, and also because the stitching weights vary depending on $N_{\pileup}$.
Instead, we show in Fig.~\ref{fig:weight_trigger_rate} the spectrum of the stitching weights
that we obtain when inserting the numbers given in Tables~\ref{tab:samples_trigger_rate} and~\ref{tab:p_trigger_rate} into Eq.~(\ref{eq:weight_trigger_rate_simplified}).
For comparison, we also show the corresponding weight, given by $s_{\incl} = F/N_{\incl}$,
for the case that only the inclusive sample is used to estimate the trigger rate.
The weight $s_{\incl}$ amounts to $35$~Hz in this example.
As can be seen in Fig.~\ref{fig:weight_trigger_rate}, the addition of samples produced in ranges in $\pThat$ to the inclusive sample reduces the weights.
Lower weights reduce the statistical uncertainties on the estimate of the trigger rate.
The different maxima in the distribution of stitching weights $s^{I}$ correspond to events 
in which the transverse momentum exchanged between the scattered protons falls into different ranges in $\pThat$.
The spectrum of weights shown in Fig.~\ref{fig:weight_trigger_rate} is plotted before any trigger selection is applied.
The stitching weights $s^{I}$ are on average smaller for events that pass than for events that fail the trigger selection,
as the probability for an event to pass the trigger increases with $\pThat$.
In contrast, the weight $s_{\incl}$ is the same before and after the trigger selection is applied.
Consequently, the reduction in statistical uncertainties that one obtains by using the exclusive samples and applying the stitching weights 
becomes even more pronounced after the trigger selection is applied.

\begin{figure}
\setlength{\unitlength}{1mm}
\begin{center}
\includegraphics*[height=76mm]{plots/makeEvtWeightPlotsForPaper_evtWeight_log.pdf}
\end{center}
\caption{
  Stitching weights $s^{I}$, computed according to Eq.~(\ref{eq:weight_trigger_rate_simplified}), 
  for the inclusive sample and for the samples produced in ranges of $\pThat$.
}
\label{fig:weight_trigger_rate}
\end{figure}

The rates expected for a single jet trigger and for a dijet trigger at the HL-LHC are shown in Fig.~\ref{fig:trigger_rate}.
The rates are computed as function of the $\pT$ threshold that is applied to the jets. 
In case of the dijet trigger, the same $\pT$ threshold is applied to both jets.
The jets are reconstructed as described in Section~\ref{sec:WJets_vs_Njet}
and are required to be within the geometric acceptance $\vert\eta\vert < 5.0$.
The rate estimate obtained for the inclusive sample and for the sum of inclusive plus exclusive samples, 
with the stitching weights computed according to Eq.~(\ref{eq:weight_trigger_rate_simplified}),
agree within statistical uncertainties, demonstrating that the estimate of the trigger rate obtained from the stitching procedure is unbiased.
The statistical uncertainties on the rate estimates obtained from the inclusive sample are represented by error bars,
while those obtained from the sum of inclusive plus exclusive samples are represented by the shaded area.

\begin{figure}
\setlength{\unitlength}{1mm}
\begin{center}
\begin{picture}(180,86)(0,0)
\put(4.5, 4.0){\mbox{\includegraphics*[height=86mm]
  {plots/makeRatePlotsForPaper_SingleJet_absEtaLt5p00_log.pdf}}}
\put(83.5, 4.0){\mbox{\includegraphics*[height=86mm]
  {plots/makeRatePlotsForPaper_DoubleJet_absEtaLt5p00_log.pdf}}}
\put(43.0, 0.0){\small (a)}
\put(122.0, 0.0){\small (b)}
\end{picture}
\end{center}
\caption{
  Rate expected for (a) a single jet trigger and (b) a dijet trigger at the HL-LHC, as function of the $\pT$ threshold that is applied to the jets.
}
\label{fig:trigger_rate}
\end{figure}

The statistical uncertainties obtained with the stitching procedure are smaller than the ones obtained in case only the inclusive sample is used.
The reduction in the statistical uncertainties is more pronounced for the dijet trigger than for the single jet trigger.
The single jet trigger rate decreases only modestly for jet $\pT$ thresholds greater than $400$~\GeV.
This ``flattening'' of the trigger rate is due to events with low $\pThat$ that contain a single jet of high $\pT$.
The stitching weights $s^{I}$ for these low $\pThat$ events are not much smaller than the weights for the inclusive sample,
which explains why the statistical uncertainties on the single jet trigger rate are reduced only modestly by the stitching procedure.
Thr requirement of a second high $\pT$ jet removes most of these low $\pThat$ events.
This has the effect that the dijet trigger rate decreases more rapidly as function of the jet $\pT$ threshold.
And the stitching procedure becomes more effective in reducing the statistical uncertainties on the trigger rate.
