\section{Examples}
\label{sec:examples}

In this Section, we illustrate the formalism developed in Section~\ref{sec:stitching_weights} with concrete examples,
drawn from two different applications: the estimation of 
%Drell-Yan (DY) and 
$\PW$+jets backgrounds in physics analyses at the LHC
and the estimation of trigger rates for the high-luminosity LHC (HL-LHC) upgrade~\cite{TDR_Phase2_LHC},
scheduled to start operation in 2027.

The 
%DY production of lepton pairs ($\PZggx \to \Plepton\Plepton$) as well as the 
production of $\PW$ bosons with subsequent decay to a charged lepton and a neutrino ($\PW \to \Plepton\Pnu$)
%constitute 
constitutes a relevant backgrounds to many physics analyses at the LHC,
for example the analysis of SM Higgs ($\PHiggs$) boson production in the decay 
%modes $\PHiggs \to \Pgt\Pgt$ and 
mode $\PHiggs \to \PW\PW$
and the search for $\PHiggs$ boson pair production in the decay 
%modes $\PHiggs\PHiggs \to \Pbottom\Pbottom\Pgt\Pgt$ and 
%$\PHiggs\PHiggs \to \Pbottom\Pbottom\PW\PW$~\cite{ATLAS:2014aga,Aad:2015vsa,Aad:2019yxi,Aaboud:2018sfw,CMS-HIG-13-004,CMS-HIG-13-027,CMS-HIG-17-002,CMS-HIG-17-006}.
mode $\PHiggs\PHiggs \to \Pbottom\Pbottom\PW\PW$~\cite{ATLAS:2014aga,Aad:2019yxi,CMS-HIG-13-027,CMS-HIG-17-006}.
Simulated samples of $\PW$+jets 
%and DY 
events have been produced for $\Pp\Pp$ collisions at $\sqrt{s}=13$~\TeV center-of-mass energy
using matrix elements computed at leading order (LO) 
%and at next-to-leading order (NLO) 
accuracy in perturbative quantum chromodynamics (pQCD)
with the program \MGvATNLO $2.4.2$~\cite{Alwall:2014hca}.
%The production of DY events is restricted to the fiducial region $m_{\Plepton\Plepton} > 50$~\GeV, where $m_{\Plepton\Plepton}$ denotes the mass of the lepton pair.
Parton showering, hadronization, and the underlying event are modeled using the program \PYTHIA $v8.2$~\cite{Sjostrand:2014zea} with the tune \textrm{CP5}~\cite{Sirunyan:2019dfx}.
The matching of matrix elements to parton showers is done using the \textrm{MLM} scheme~\cite{Alwall:2007fs}.% for the LO samples
%and the \textrm{FXFX} scheme~\cite{Frederix:2012ps} for the NLO samples.
The 
%DY and $\PW$+jets 
samples are normalized using cross sections computed at next-to-next-to leading order (NNLO) accuracy in pQCD,
with electroweak corrections taken into account up to NLO accuracy~\cite{Li:2012wna}.
%The product of the DY cross section in the PS region $m_{\Plepton\Plepton} > 50$~\GeV
%times the branching fraction for the decay into two charged leptons amounts to $6.08$~nb,
%while the cross section for $\PW$+jets production times the branching fraction for the decay to charged lepton and neutrino amounts to $61.5$~nb.
The product of the $\PW$+jets production cross section times the branching fraction for the decay to a charged lepton and a neutrino amounts to $61.5$~nb.

We will demonstrate the stitching of these samples based on two observables,
$N_{\jet}$ and $\HT$, defined as, respectively, the number of jets and the scalar sum in $\pT$ of jets in the event.
The PS region in which we perform the stitching will be either one- or two-dimensional.
We will show that for our formalism
it makes little difference whether the stitching is performed in one dimension or in two:
The regions in PS are enumerated by a single index $i$ in either case,
and in either case the probability $P^{i}$ follows a categorical distribution.

The task of estimating trigger rates for the upcoming high-luminosity data-taking period of the LHC is chosen as second example to illustrate the stitching procedure.
The ``rate'' of a trigger corresponds to the number of $\Pp\Pp$ collision events that satisfy the trigger condition per unit of time.
The estimation of trigger rates constitutes an important task for demonstrating the physics potential of the HL-LHC.
The HL-LHC physics program demands a large amount of integrated luminosity to be delivered by the LHC, 
in order to facilitate measurements of rare signal processes,
such as the precise measurement of $\PHiggs$ boson couplings and the study of $\PHiggs$ boson pair production, by the ATLAS and CMS experiments.
In order to satisfy this demand, the HL-LHC is expected to operate at an instantaneous luminosity of $5$-$7.5 \times 10^{34}$~cm$^{-2}$~s$^{-1}$
at a center-of-mass energy of $\sqrt{s} = 14$~\TeV~\cite{TDR_Phase2_LHC}.
The challenge of developing triggers for the HL-LHC is to design the triggers such that rare signal processes pass the triggers with a high efficiency,
while the rate of background processes gets reduced by many orders of magnitude, in order not to exceed bandwidth limitations on the detector read-out 
and on the rate with which events can be written to permanent storage.

The inelastic $\Pp\Pp$ scattering cross section at $\sqrt{s} = 14$~\TeV amounts to $\approx 80$~mb,
resulting in up to $200$ simultaneous $\Pp\Pp$ interactions per crossing of the proton beams at the nominal HL-LHC instantaneous luminosity~\cite{TDR_Phase2_LHC}.
The vast majority of these interactions are inelastic $\Pp\Pp$ scatterings with low momentum exchange,
which predominantly arise from the exchange of gluons between the colliding protons.
We refer to inelastic $\Pp\Pp$ scattering interactions with no further selection applied as ``minimum bias'' events.
In order to estimate the rates of triggers at the HL-LHC,
MC samples of minimum bias events are produced at LO in pQCD using the program \PYTHIA.
The minimum bias samples are complemented by samples of inelastic $\Pp\Pp$ scattering interactions
in which a significant amount of transverse momentum, denoted by the symbol $\pThat$, is exchanged between the scattered protons.
The stitching of the minimum bias samples with samples generated for different ranges in $\pThat$ allows to estimate the trigger rates with lower statistical uncertainties.

The production of MC samples used for estimating trigger rates at the HL-LHC
proceeds by first simulating one ``hard-scatter'' (HS) interaction within a given range in $\pThat$
and then adding a number of additional inelastic $\Pp\Pp$ scattering interactions of the minimum bias kind to the same event.
We refer to these additional inelastic $\Pp\Pp$ scattering interactions as ``pileup'' (PU)
and use the symbol $N_{\pileup}$ to denote the total number of these additional inelastic $\Pp\Pp$ scattering interactions 
that occur in the same crossing of the proton beams as the HS interaction.
No selection on $\pThat$ is applied when simulating the PU interactions.
The distinction between the HS interaction and the PU interactions is artificial and solely made for the purpose of MC production.
The HS interaction and the PU interactions will be indistinguishable in the data that will be recorded at the HL-LHC.
The scattering in which the transverse momentum exchange between the protons amounts to $\pThat$ may occur in any of the $N_{\pileup} + 1$ simultaneous $\Pp\Pp$ interactions.
Our formalism treats the HS interaction and the $N_{\pileup}$ additional PU interactions on an equal footing.
We enumerate the regions in PS of the HS and of the PU interactions by a vector $I$ of dimension $N_{\pileup} + 1$.
The $k$-th component of this vector indicates the range in $\pThat$ of the $k$-th $\Pp\Pp$ interaction.
The probability $P^{I} = P^{i_{1},\dots,i_{N_{\pileup}+1}}$ follows a multinomial distribution.


