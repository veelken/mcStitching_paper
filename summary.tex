\section{Summary}
\label{sec:summary}

The production of MC samples of adequate size is often a challenge in modern HEP experiments,
due to the computing resources required to produce and store such samples.
This is particularly true for experiments at the CERN LHC,
firstly because of the large $\Pp\Pp$ scattering cross section and secondly because of the large luminosity delivered by the LHC.

In this paper we have presented a procedure that allows to reduce the statistical uncertainties 
by combining MC samples which overlap in PS.
Our formalism is general enough to be applied in various use-cases.
Different examples for applying the formalism to $\Pp\Pp$ collisions at the LHC are given in this paper.

Of particular interest is the case of modelling background contributions to searches for new physics,
where potential signals are typically expected to be small and often similar in size to the statistical uncertainties on the background contributions.
The statistical uncertainties on these background contributions can often be significantly reduced 
by dividing the PS into multiple regions, producing separate MC samples for each region,
and accounting for the overlap of different samples in PS by applying the weights computed as detailed in this paper to the simulated events.
We refer to this procedure as ``stitching''.

The flexibility of our formalism is demonstrated by applying the stitching procedure to the case of estimating trigger rates at the HL-LHC,
where up to $200$ simultaneous $\Pp\Pp$ collisions are expected per crossing of the proton beams.
