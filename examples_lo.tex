\subsection{Modelling of \texorpdfstring{$\PW$}{W}+jets production in physics analyses at the LHC}
\label{sec:examples_background_yield}

The production of $\PW$ bosons is interesting to study at the LHC for several reasons.
The measurement of the mass of the $\PW$ boson is an important input to global fits to SM parameters~\cite{Baak:2014ora}.
The fits allow to test the overall consistency of the SM and to set constraints on physics beyond the SM.
The sensitivity of these fits is currently limited by the precision of the $\PW$ boson mass measurement~\cite{Baak:2014ora}.
Differential measurements of the cross section for $\PW$+jets production 
are used to constrain parton distribution functions~\cite{CMS:2016qqr,ATLAS:2016nqi,ATLAS:2019fgb,CMS:2020cph}.
In particular, the measurement of the associated production of a $\PW$ boson with a charm quark
provides sensitivity to the strange quark content of the proton~\cite{CMS:2013wql,ATLAS:2014jkm,CMS:2018dxg} 
and allows to tune MC generators to improve the modelling of heavy flavour production at hadron colliders.
The production of $\PW$ bosons also constitutes a relevant background to measurements of other SM processes
and to searches for new physics, see for example Refs.~\cite{ATLAS:2014aga,Aad:2019yxi,CMS-HIG-13-027,CMS-HIG-17-006}.
In this section, we focus on $\PW$+jets production with subsequent leptonic decay of the $\PW$ boson.

Simulated samples of $\PW$+jets events have been produced for $\Pp\Pp$ collisions at $\sqrt{s}=13$~\TeV center-of-mass energy
using matrix elements computed at leading order (LO) accuracy in perturbative quantum chromodynamics (pQCD)
with the program \MGvATNLO $2.6.5$~\cite{MGvATNLO}.
The parton distribution functions of the proton are modeled using the NNPDF3.1 set~\cite{NNPDF:2017mvq}.
Parton showering, hadronization, and the underlying event are modeled using the program \PYTHIA $v8.240$~\cite{PYTHIA} with the tune \textrm{CP5}~\cite{Sirunyan:2019dfx}.
The matching of matrix elements to parton showers is done using the \textrm{MLM} scheme~\cite{Alwall:2007fs}.
We restrict the analysis of these samples to particles originating from the hard-scatter interaction and do not add any pileup to these samples.
Samples containing either $1$, $2$, $3$, or $4$ jets at matrix-element level are complemented by an ``inclusive'' sample 
and by samples binned in the scalar sum in $\pT$ of these jets.
We denote the multiplicity of jets at the matrix-element level by the symbol $N_{\jet}$ and the scalar sum in $\pT$ of these jets by the symbol $\HT$.
The inclusive and $\HT$-binned samples contain events with between $0$ and $4$ jets at the matrix-element level.

The weights $w_{j}^{k}$ and $w_{\incl}^{k}$ are equal to one for all events in these samples.
Thus, $\sum_{k=1}^{N_{j}} \, w_{j}^{k} = N_{j}$ and $\sum_{k=1}^{N_{\incl}} \, w_{\incl}^{k} = N_{\incl}$ for this example,
which allows us to simplify Eq.~(\ref{eq:weight_incl}) to:
\begin{equation}
s^{i} = \frac{L \, \sigma_{\incl}}{N_{\incl}} \, \frac{P_{\incl}^{i} \, N_{\incl}}{P_{\incl}^{i} \, N_{\incl} + P_{j}^{i} \, N_{j}} \, .
\label{eq:weight_incl_simplified}
\end{equation}

All samples are normalized using a $k$-factor of $1.14$, given by the ratio of the inclusive cross section computed at next-to-next-to leading order (NNLO) accuracy in pQCD,
with electroweak corrections taken into account up to NLO accuracy~\cite{Li:2012wna},
and the inclusive cross section computed at LO accuracy by the program \MGvATNLO.
The product of the inclusive $\PW$+jets production cross section times the branching fraction for the decay to a charged lepton and a neutrino amounts to $61.5$~nb.

We will demonstrate the stitching of these samples based on the two observables $N_{\jet}$ and $\HT$.
The PS region in which we perform the stitching will be either one- or two-dimensional.
We will show that for our formalism
it makes little difference whether the stitching is performed in one dimension or in two.
The stitching of $\PW$+jets samples based on the observable $N_{\jet}$ will be discussed first
and then we will discuss the stitching of $\PW$+jets samples based on the two observables $N_{\jet}$ and $\HT$.


\subsubsection{Stitching of \texorpdfstring{$\PW$}{W}+jets samples by \texorpdfstring{$N_{\jet}$}{Njet}}
\label{sec:WJets_vs_Njet}

In this example, an inclusive $\PW$+jets sample simulated at LO accuracy in pQCD 
is stitched with exclusive samples containing events with $N_{\jet}$ equal to either $1$, $2$, $3$, or $4$.
The inclusive sample contains events with $N_{\jet}$ between $0$ and $4$.
We partition the PS into slices based on the multiplicity of jets at the matrix-element level and set the index $i$ equal to $N_{\jet}$.
The number of events in each MC sample is chosen such that the stitching weights decrease by about a factor of two for each increase in jet multiplicity.
The decrease in the cross section as function of $N_{\jet}$
allows to reduce the statistical uncertainties in the tail of the $N_{\jet}$ distribution
without significantly increasing the expenditure of computing resources required to produce and store these samples.
The number of events contained in each sample and the values of the probabilities $P_{\incl}^{i}$ and $P_{j}^{i}$ are given in Table~\ref{tab:samples_and_probabilities_WJets_vs_Njet}.
The probabilities $P_{\incl}^{1}$, $P_{\incl}^{2}$, $P_{\incl}^{3}$, and $P_{\incl}^{4}$ are computed by taking the ratio of cross sections,
computed at LO accuracy by the program \MGvATNLO,
for the exclusive samples with respect to the cross section $\sigma_{\incl}$ of the inclusive sample.
The probability $P_{\incl}^{0}$ is obtained using the relation $P_{\incl}^{0} = 1 - \sum_{i=1}^{4} P_{\incl}^{i}$.
The probabilities $P_{j}^{i}$ for the exclusive samples are $1$ if $i=j$ and $0$ otherwise,
as each of the exclusive samples $j$ covers exactly one PS region $i$.
The corresponding stitching weights $s^{i}$, computed according to Eq.~(\ref{eq:weight_incl_simplified}), are given in Table~\ref{tab:weights_WJets_vs_Njet}.

In order to demonstrate that the stitching procedure is unbiased,
we compare the normalization and shape of distributions obtained using the stitching procedure with the normalization and shape of distributions obtained from the inclusive sample.
Distributions in $\pT$ of the ``leading'' and ``subleading'' jet (the jets of, respectively, highest and second-highest $\pT$ in the event),
in the multiplicity of jets and in the observable $\HT$ are shown in Fig.~\ref{fig:controlPlots_WJets_vs_Njet}.
The distributions obtained from the inclusive sample are represented by black markers (``inclusive only''),
while those obtained by applying the stitching procedure to the combination of the inclusive sample and the samples binned in $N_{\jet}$ are represented by pink lines (``stitched'').
For illustration,
the contributions of individual exclusive samples $j$ to the stitched distribution are represented by different colors and fill patterns in the upper part of each figure.
The contribution of the inclusive sample to the stitched distribution is represented by the white area (``inclusive stitched'').
We remark that the ``inclusive only'' and ``inclusive stitched'' distributions contain the exact same events. 
The sole difference between these two distributions 
is that the stitching weights, given in Table~\ref{tab:weights_WJets_vs_Njet},
are applied to the ``inclusive stitched'', but not to the ``inclusive only'' distribution.
In the lower part of each figure, we show the difference in normalization and shape between the distribution obtained using the stitching procedure
and the distribution obtained when using solely the inclusive sample.
The differences are given relative to the distribution obtained from our stitching procedure.
The size of statistical uncertainties on the ``inclusive only'' and ``stitched'' distributions
is visualized in the lower part of each figure and is represented by the length of the error bars and by the height of the dark shaded area, respectively.
The jets shown in the figure are reconstructed using the anti-\kt algorithm~\cite{Cacciari:2008gp,Cacciari:2011ma} with a distance parameter of $0.4$,
using all stable generator-level particles (after hadron shower and hadronization) except neutrinos as input, and are required to satisfy the selection criteria $\pT > 25$~\GeV and $\vert\eta\vert < 5.0$.
The observable $\HT$ is computed as the scalar sum in $\pT$ of these jets.
Note that the multiplicity of jets and the observable $\HT$ shown in Fig.~\ref{fig:controlPlots_WJets_vs_Njet} 
differ from the observables $N_{\jet}$ and $\HT$ that are used in the stitching procedure:
The former refer to jets at the generator (detector) level, while the latter refer to jets at the matrix-element level.
The distributions are normalized to an integrated luminosity of $140$~fb$^{-1}$.

The distributions for the inclusive sample and for the sum of inclusive plus exclusive samples, with the stitching weights applied, are in agreement within the statistical uncertainties.
The exclusive samples reduce the statistical uncertainties in particular in the tails of the distributions.

\begin{table}[h!]
\begin{center}
\def\arraystretch{1.3}
\begin{tabular}{l|c|c|c|ccccc}
\multirow{2}{20mm}{Sample} & Index & Number    & Cross                    & \multicolumn{5}{c}{Probabilities}               \\
                           & $j$   & of events & section [nb]$^{\dagger}$ & $P^{0}$ & $P^{1}$ & $P^{2}$ & $P^{3}$ & $P^{4}$ \\
\hline
Inclusive                  & $-$   & $3 \times 10^{6}$ & $61.5$ & $0.758$ & $0.167$ & $0.052$ & $0.015$ & $0.007$ \\
$N_{\jet} = 1$             & $1$   & $5 \times 10^{6}$   & $10.1$  & $0$     & $1$     & $0$     & $0$     & $0$  \\
$N_{\jet} = 2$             & $2$   & $4.7 \times 10^{6}$ & $3.21$  & $0$     & $0$     & $1$     & $0$     & $0$  \\
$N_{\jet} = 3$             & $3$   & $3.2 \times 10^{5}$ & $0.938$ & $0$     & $0$     & $0$     & $1$     & $0$  \\
$N_{\jet} = 4$             & $4$   & $3.3 \times 10^{5}$ & $0.443$ & $0$     & $0$     & $0$     & $0$     & $1$  \\
\end{tabular}
\end{center}
$^{\dagger}$ Computed at LO accuracy in pQCD, then scaled to NNLO
\caption{
  Number of events in the inclusive $\PW$+jets sample and in the $\PW$+jets samples produced in bins of $N_{\jet}$,
  corresponding cross sections,
  and probabilities $P^{i}$ for the events in the inclusive and exclusive samples to populate the different PS regions $i$.
}
\label{tab:samples_and_probabilities_WJets_vs_Njet}
\end{table}

\begin{table}[h!]
\begin{center}
\begin{tabular}{l|ccccc}
 & \multicolumn{5}{c}{Multiplicity of jets} \\
 & $0$ & $1$ & $2$ & $3$ & $4$ \\
\hline
Stitching weight & $2870$ & $1440$ & $714$ & $362$ & $180$ \\
\end{tabular}
\end{center}
\caption{
  Stitching weights $s^{i}$ for the case that the inclusive and exclusive $\PW$+jets samples 
  given in Table~\ref{tab:samples_and_probabilities_WJets_vs_Njet}
  are stitched based on $N_{\jet}$.
  The weights are computed for an integrated luminosity of $140\fbinv$.
}
\label{tab:weights_WJets_vs_Njet}
\end{table}

\begin{figure}
\setlength{\unitlength}{1mm}
\begin{center}
\begin{picture}(180,182)(0,0)
\put(6.5, 100.0){\mbox{\includegraphics*[height=82mm]{plots/WJets_Njet_lead_stack_wRatio_log.pdf}}}
\put(81.5, 100.0){\mbox{\includegraphics*[height=82mm]{plots/WJets_Njet_sublead_stack_wRatio_log.pdf}}}
\put(6.5, 4.0){\mbox{\includegraphics*[height=82mm]{plots/WJets_Njet_njet_stack_wRatio_log.pdf}}}
\put(81.5, 4.0){\mbox{\includegraphics*[height=82mm]{plots/WJets_Njet_ht_stack_wRatio_log.pdf}}}
\put(43.0, 96.0){\small (a)}
\put(118.0, 96.0){\small (b)}
\put(43.0, 0.0){\small (c)}
\put(118.0, 0.0){\small (d)}
\end{picture}
\end{center}
\caption{
  Distributions in $\pT$ of the (a) leading and (b) subleading jet,
  in (c) the multiplicity of generator-level jets and in (d) the observable $\HT$, the scalar sum in $\pT$ of these jets,
  for the case of $\PW$+jets samples that are stitched based on the observable $N_{\jet}$ at the matrix-element level.
  The $\PW$ bosons are required to decay leptonically.
  The event yields are computed for an integrated luminosity of $140\fbinv$.
}
\label{fig:controlPlots_WJets_vs_Njet}
\end{figure}


\subsubsection{Stitching of \texorpdfstring{$\PW$}{W}+jets samples by \texorpdfstring{$N_{\jet}$}{Njet} and \texorpdfstring{$\HT$}{HT}}
\label{sec:WJets_vs_Njet_and_HT}

This example extends the previous example.
It demonstrates the stitching procedure based on two observables, $N_{\jet}$ and $\HT$.
The exclusive samples are simulated for jet multiplicities of $N_{\jet} = 1$, $2$, $3$, and $4$ 
and for $\HT$ in the ranges $70$-$100$, $100$-$200$, $200$-$400$, $400$-$600$, $600$-$800$, $800$-$1200$, $1200$-$2500$, and $> 2500$~\GeV (up to the kinematic limit).
We refer to the exclusive samples produced in slices of $N_{\jet}$ as the ``$N_{\jet}$-samples''
and to the samples simulated in ranges in $\HT$ as the ``$\HT$-samples''.
The inclusive sample contains events with jet multiplicities between $0$ and $4$ and covers the full range in $\HT$.
The number of events in the $\HT$-samples are given in Table~\ref{tab:samples_WJets_vs_Njet_and_HT}.
The information for the inclusive sample and for the $N_{\jet}$-samples is the same as for the previous example
and is given in Table~\ref{tab:samples_and_probabilities_WJets_vs_Njet}.

The corresponding PS regions $i$, defined in the plane of $N_{\jet}$ versus $\HT$, are shown in Fig.~\ref{fig:regions_WJets_vs_Njet_and_HT}.
In total, the probabilities $P_{\incl}^{i}$ and $P_{j}^{i}$ and the corresponding stitching weights $s^{i}$ are computed for $45$ separate PS regions.

In some of the $45$ PS regions, the probabilities $P_{\incl}^{i}$ are rather low, on the level of $10^{-7}$.
In order to reduce the statistical uncertainties on the probabilities $P_{\incl}^{i}$ and $P_{j}^{i}$,
we compute these probabilities using the following procedure:
For all regions $i$ of PS that are covered by the inclusive sample and by one or more $N_{\jet}$- or $\HT$-samples,
we determine the probabilities $P_{\incl}^{i}$ and $P_{j}^{i}$ by the method of least squares~\cite{Cowan:1998ji}.
Details of the computation are given in the appendix.
The probability $P_{\incl}^{0}$ for the PS region $N_{\jet} = 0$ and $\HT < 70$~\GeV,
which is solely covered by the inclusive sample,
is computed according to the relation $P_{\incl}^{0} = 1 - \sum_{i=1}^{44} \, P_{\incl}^{i}$.
The numerical values of the probabilities $P_{\incl}^{i}$ and $P_{j}^{i}$ obtained by the least-square method
and of the stitching weights $s^{i}$, computed according to Eq.~(\ref{eq:weight_incl_simplified}), are given in Tables~\ref{tab:probabilities_WJets_vs_Njet_and_HT}
and~\ref{tab:weights_WJets_vs_Njet_and_HT} in the appendix.

Distributions in $\pT$ of the leading and subleading jet,
in the multiplicity of jets, and in the observable $\HT$ 
obtained from our stitching procedure are compared to the distributions obtained from the inclusive sample in Fig.~\ref{fig:controlPlots_WJets_vs_Njet_and_HT}.
The distributions obtained by stitching the inclusive sample with the samples binned in $N_{\jet}$ and in $\HT$ are represented by pink lines,
while those obtained when using solely the inclusive sample are represented by black markers.
The weights given in Table~\ref{tab:weights_WJets_vs_Njet_and_HT} are applied to the stitched distributions.
The contribution of all $N_{\jet}$-binned samples to the stitched distribution is represented by the blue shaded area (``sum of $N_{\jet}$ samples''),
while the contribution of all samples produced in ranges in $\HT$ is represented by the yellow shaded area (``sum of $\HT$ samples'') in the upper part of the figures.
Following Fig.~\ref{fig:controlPlots_WJets_vs_Njet},
the white area represents the contribution of the inclusive sample to the stitched distribution.
The jets are reconstructed as described in Section~\ref{sec:WJets_vs_Njet} and are required to pass the selection criteria $\pT > 25$~\GeV and $\vert\eta\vert < 5.0$.
In the lower part of each figure, we again show the difference between the distributions obtained from the inclusive sample and obtained by using our stitching procedure,
and also the respective statistical uncertainties.
As one would expect, the addition of samples simulated in ranges in $\HT$ to the example given in Section~\ref{sec:WJets_vs_Njet}
reduces the statistical uncertainties in the tails of the distributions.
The reduction is most pronounced in the tails of the distributions in leading and subleading jet $\pT$ and in the observable $\HT$.

\begin{table}[h!]
\begin{center}
\def\arraystretch{1.3}
\begin{tabular}{l|c|c|c}
\multirow{2}{20mm}{Sample}       & Index & Number    & Cross                    \\
                                 & $j$   & of events & section [pb]$^{\dagger}$ \\
\hline
$  70 \leqslant \HT <  100$~\GeV &  $5$  & $1.4 \times 10^{5}$ & $1430$  \\
$ 100 \leqslant \HT <  200$~\GeV &  $6$  & $2.8 \times 10^{5}$ & $1410$  \\
$ 200 \leqslant \HT <  400$~\GeV &  $7$  & $1.5 \times 10^{5}$ & $379$   \\
$ 400 \leqslant \HT <  600$~\GeV &  $8$  & $  4 \times 10^{4}$ & $51.3$  \\
$ 600 \leqslant \HT <  800$~\GeV &  $9$  & $  2 \times 10^{4}$ & $12.6$  \\
$ 800 \leqslant \HT < 1200$~\GeV & $10$  & $  2 \times 10^{4}$ & $5.34$  \\
$1200 \leqslant \HT < 2500$~\GeV & $11$  & $1.3 \times 10^{4}$ & $1.51$  \\
$       \HT \geqslant 2500$~\GeV & $12$  & $6.5 \times 10^{2}$ & $0.052$ \\
\end{tabular}
\end{center}
$^{\dagger}$ Computed at LO accuracy in pQCD, then scaled to NNLO
\caption{
  Number of events in the $\PW$+jets samples produced in ranges in $\HT$ and corresponding cross sections.
}
\label{tab:samples_WJets_vs_Njet_and_HT}
\end{table}

\begin{figure}
\setlength{\unitlength}{1mm}
\begin{center}
\begin{picture}(180,82)(0,0)
\includegraphics*[height=82mm]{plots/regions_WJets_vs_Njet_and_HT.pdf}
\end{picture}
\end{center}
\caption{
  Definition of the PS regions $i$ in the plane of $N_{\jet}$ versus $\HT$,
  for the case of $\PW$+jets samples that are stitched based on the observables $N_{\jet}$ and $\HT$.
}
\label{fig:regions_WJets_vs_Njet_and_HT}
\end{figure}

\begin{figure}
\setlength{\unitlength}{1mm}
\begin{center}
\begin{picture}(180,182)(0,0)
\put(6.5, 100.0){\mbox{\includegraphics*[height=82mm]{plots/WJets_lead_stack_wRatio_log.pdf}}}
\put(81.5, 100.0){\mbox{\includegraphics*[height=82mm]{plots/WJets_sublead_stack_wRatio_log.pdf}}}
\put(6.5, 4.0){\mbox{\includegraphics*[height=82mm]{plots/WJets_njet_stack_wRatio_log.pdf}}}
\put(81.5, 4.0){\mbox{\includegraphics*[height=82mm]{plots/WJets_ht_stack_wRatio_log.pdf}}}
\put(43.0, 96.0){\small (a)}
\put(118.0, 96.0){\small (b)}
\put(43.0, 0.0){\small (c)}
\put(118.0, 0.0){\small (d)}
\end{picture}
\end{center}
\caption{
  Distributions in $\pT$ of the (a) leading and (b) subleading jet,
  in (c) the multiplicity of generator-level jets and in (d) the observable $\HT$, the scalar sum in $\pT$ of these jets,
  for the case of $\PW$+jets samples that are stitched based on the observables $N_{\jet}$ and $\HT$ at the matrix-element level.
  The $\PW$ bosons are required to decay leptonically.
  The event yields are computed for an integrated luminosity of $140\fbinv$.
}
\label{fig:controlPlots_WJets_vs_Njet_and_HT}
\end{figure}


