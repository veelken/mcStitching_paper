%\subsection{Estimation of DY and \texorpdfstring{$\PW$}{W}+jets backgrounds}
\subsection{Estimation of \texorpdfstring{$\PW$}{W}+jets backgrounds}
\label{sec:examples_background_yield}

The examples in this section refer to the modelling of 
%DY and 
$\PW$+jets backgrounds in the context of physics analyses at the LHC.
We will first discuss the stitching of $\PW \to \Plepton\Pnu$ samples
%generated at LO accuracy in pQCD 
based on the observable $N_{\jet}$
%, followed by a discussion of the stitching of $\PW \to \Plepton\Pnu$ samples generated at LO accuracy in pQCD based on the two observables $N_{\jet}$ and $\HT$,
%before we conclude this section with a discussion of stitching $\PZggx \to \Plepton\Plepton$ samples generated at NLO in pQCD based on the multiplicity of jets.
and then proceed to discuss the stitching of $\PW \to \Plepton\Pnu$ samples based on the two observables $N_{\jet}$ and $\HT$.
In 
%all three 
both cases, we will assume that an inclusive sample, covering the whole PS, is available.


%\subsubsection{Stitching of LO \texorpdfstring{$\PW$}{W}+jets samples by \texorpdfstring{$N_{\jet}$}{Njet}}
\subsubsection{Stitching of \texorpdfstring{$\PW$}{W}+jets samples by \texorpdfstring{$N_{\jet}$}{Njet}}
\label{sec:WJets_vs_Njet}

In this example, an inclusive $\PW \to \Plepton\Pnu$ sample simulated at LO accuracy in pQCD 
is stitched with exclusive samples produced for jet multiplicities of $N_{\jet} = 1$, $2$, $3$, and $4$.
The inclusive sample contains events with jet multiplicities between $0$ and $4$.
We divide the PS by the number of jets and set the index $i$ equal to $N_{\jet}$.
The number of events in each MC sample and the values of the $P^{i}$ and $P_{j}^{i}$ are given in Table~\ref{tab:samples_and_probabilities_WJets_vs_Njet}.
The probabilities $P^{1}$,$\ldots$,$P^{4}$ are computed by taking the ratio of cross sections 
for the $N_{\jet} = 1$,$\ldots$,$4$ samples with respect to the cross section $\sigma_{\incl}$ of the inclusive sample.
The cross sections used for computing these ratios have been calculated at LO accuracy in pQCD using the program \MGvATNLO
and have been upgraded to NNLO accuracy by scaling all cross sections by the ratio ($k$-factor) of the NNLO to LO inclusive cross sections.
The probability $P^{0}$ is obtained using the relation $P^{0} = 1 - \sum_{i=1}^{4} P^{i}$.
The probabilities $P_{j}^{i}$ for the exclusive samples are $1$ if $i=j$ and $0$ otherwise,
as each of the exclusive samples $j$ covers exactly one of the PS regions $i$.
The corresponding stitching weights, computed according to Eq.~(\ref{eq:weight_incl}), are given in Table~\ref{tab:weights_WJets_vs_Njet}.

Except for the $N_{\jet} = 3$ and $N_{\jet} = 4$ regions,
the weights $w^{i}$ decrease as the number of jets increases, 
reflecting the reduction in statistical uncertainty that is achieved by using the exclusive samples in combination with the inclusive one.
The weights $w^{3}$ and $w^{4}$ for the $N_{\jet} = 3$ and $N_{\jet} = 4$ regions are about the same,
reflecting the fact that the number of events in the $N_{\jet} = 4$ sample is smaller compared to the number of events in the $N_{\jet} = 3$ sample
by about the same factor as the ratio of the corresponding cross sections.

In order to demonstrate that the stitching procedure yields background estimates that are unbiased,
we show distributions in $\pT$ of the ``leading'' and ``subleading'' jet (the jets of, respectively, highest and second-highest $\pT$ in the event),
in the multiplicity of jets and in the observable $\HT$ 
for the inclusive sample and for the sum of inclusive plus exclusive samples in Fig.~\ref{fig:controlPlots_WJets_vs_Njet}.
Jets are reconstructed using the anti-\kt algorithm~\cite{Cacciari:2008gp,Cacciari:2011ma} with a distance parameter of $0.4$,
using all stable particles except neutrinos as input, and are required to satisfy the selection criteria $\pT > 25$~\GeV and $\vert\eta\vert < 5.0$.
The distributions are normalized to an integrated luminosity of $140$~fb$^{-1}$, recorded at $\sqrt{s}=13$~\TeV.
Individual exclusive samples $j$ are distinguished by different colors and fill patterns in the upper part of each figure.
In the lower part, we show the difference between the background prediction obtained from the inclusive sample and from the sum of inclusive and exclusive samples,
using our stitching procedure.
The differences are given relative to the $\PW$+jets background estimate obtained from our stitching procedure.
The size of statistical uncertainties on the background estimates obtained from the inclusive sample and obtained from our stitching procedure
is visualized in the lower part of each figure and is represented by the the length of the error bars and by the height of the dark shaded area, respectively.

The distributions for the inclusive sample and for the sum of inclusive plus exclusive samples, with the stitching weights applied, are in agreement within the statistical uncertainties.
The exclusive samples reduce the statistical uncertainties in particular in the tails of the distributions,
which are the regions most relevant in searches for new physics.
The inclusive sample contains no events with $\HT > 1200$~\GeV. 

\begin{table}[h!]
\begin{center}
\def\arraystretch{1.3}
\begin{tabular}{l|c|c|c|ccccc}
\hline
\multirow{2}{20mm}{Sample} & Index & Number    & Cross                    & \multicolumn{5}{c}{Probabilities}               \\
                           & $j$   & of events & section [nb]$^{\dagger}$ & $P^{0}$ & $P^{1}$ & $P^{2}$ & $P^{3}$ & $P^{4}$ \\
\hline
\hline
Inclusive                  & $-$   & $3 \times 10^{6}$ & $61.5$ & $0.765$ & $0.153$ & $0.053$ & $0.019$ & $0.010$ \\
\hline
$N_{\jet} = 1$             & $1$   & $5 \times 10^{5}$ & $9.44$  & $0$     & $1$     & $0$     & $0$     & $0$     \\
$N_{\jet} = 2$             & $2$   & $3 \times 10^{5}$ & $3.25$  & $0$     & $0$     & $1$     & $0$     & $0$     \\
$N_{\jet} = 3$             & $3$   & $2 \times 10^{5}$ & $1.15$  & $0$     & $0$     & $0$     & $1$     & $0$     \\
$N_{\jet} = 4$             & $4$   & $         10^{5}$ & $0.634$ & $0$     & $0$     & $0$     & $0$     & $1$     \\
\hline
\end{tabular}
\end{center}
$^{\dagger}$ Computed at LO accuracy in pQCD, then scaled to NNLO
\caption{
  Number of events in the inclusive $\PW \to \Plepton\Pnu$ sample and in the $\PW \to \Plepton\Pnu$ samples produced in bins of jet multiplicity,
  corresponding cross sections,
  and probabilities $P^{i}$ for the events in the inclusive and exclusive samples to populate the different PS regions $i$.
}
\label{tab:samples_and_probabilities_WJets_vs_Njet}
\end{table}

\begin{table}[h!]
\begin{center}
\begin{tabular}{l|ccccc}
\hline
 & \multicolumn{5}{c}{Multiplicity of jets} \\
 & $0$ & $1$ & $2$ & $3$ & $4$ \\
\hline
\hline
Weight & $2870$ & $1380$ & $990$ & $625$ & $678$ \\
\hline
\end{tabular}
\end{center}
\caption{
  Weights $w^{i}$ for the case that the inclusive and exclusive $\PW \to \Plepton\Pnu$ samples 
  given in Table~\ref{tab:samples_and_probabilities_WJets_vs_Njet}
  are stitched based on $N_{\jet}$, the number of jets.
  The weights are computed for an integrated luminosity of $140\fbinv$.
}
\label{tab:weights_WJets_vs_Njet}
\end{table}

\begin{figure}
\setlength{\unitlength}{1mm}
\begin{center}
\begin{picture}(180,182)(0,0)
\put(6.5, 100.0){\mbox{\includegraphics*[height=82mm]{plots/WJets_Njet_lead_stack_wRatio_log.pdf}}}
\put(81.5, 100.0){\mbox{\includegraphics*[height=82mm]{plots/WJets_Njet_sublead_stack_wRatio_log.pdf}}}
\put(6.5, 4.0){\mbox{\includegraphics*[height=82mm]{plots/WJets_Njet_njet_stack_wRatio_log.pdf}}}
\put(81.5, 4.0){\mbox{\includegraphics*[height=82mm]{plots/WJets_Njet_ht_stack_wRatio_log.pdf}}}
\put(43.0, 96.0){\small (a)}
\put(118.0, 96.0){\small (b)}
\put(43.0, 0.0){\small (c)}
\put(118.0, 0.0){\small (d)}
\end{picture}
\end{center}
\caption{
  Distributions in $\pT$ of the (a) leading and (b) subleading jet,
  in (c) the multiplicity of jets and in (d) the observable $\HT$,
  %for the case that $\PW \to \Plepton\Pnu$ samples generated at LO accuracy in pQCD are stitched based on the observable $N_{\jet}$.
  for the case of $\PW \to \Plepton\Pnu$ samples that are stitched based on the observable $N_{\jet}$.
  The event yields are computed for an integrated luminosity of $140\fbinv$.
}
\label{fig:controlPlots_WJets_vs_Njet}
\end{figure}


%\subsubsection{Stitching of LO \texorpdfstring{$\PW$}{W}+jets samples by \texorpdfstring{$N_{\jet}$}{Njet} and \texorpdfstring{$\HT$}{HT}}
\subsubsection{Stitching of \texorpdfstring{$\PW$}{W}+jets samples by \texorpdfstring{$N_{\jet}$}{Njet} and \texorpdfstring{$\HT$}{HT}}
\label{sec:WJets_vs_Njet_and_HT}

This example extends the example given in Section~\ref{sec:WJets_vs_Njet}.
It demonstrates the stitching procedure based on two observables, $N_{\jet}$ and $\HT$.
The exclusive samples are simulated for jet multiplicities of $N_{\jet} = 1$, $2$, $3$, and $4$ 
and for $\HT$ in the ranges $70$-$100$, $100$-$200$, $200$-$400$, $400$-$600$, $600$-$800$, $800$-$1200$, $1200$-$2500$, and $> 2500$~\GeV (up to the kinematic limit).
We refer to the exclusive samples produced in slices of $N_{\jet}$ as the ``$N_{\jet}$-samples''
and to the samples simulated in ranges in $\HT$ as the ``$\HT$-samples''.
The inclusive sample contains events with jet multiplicities between $0$ and $4$ and covers the full range in $\HT$.
The number of events in the $\HT$-samples are given in Table~\ref{tab:samples_WJets_vs_Njet_and_HT}.
The information for the inclusive sample and for the $N_{\jet}$-samples is the same as for the previous example
and is given in Table~\ref{tab:samples_and_probabilities_WJets_vs_Njet}.

The corresponding PS regions $i$, defined in the plane of $N_{\jet}$ versus $\HT$, are shown in Fig.~\ref{fig:regions_WJets_vs_Njet_and_HT}.
In total, the probabilities $P^{i}$ and $P_{j}^{i}$ and the corresponding stitching weights $w^{i}$ are computed for $45$ separate PS regions.

In some of the $45$ PS regions, the probabilities $P^{i}$ are rather low, on the level of $10^{-7}$.
In order to reduce the statistical uncertainties on the probabilities $P^{i}$ and $P_{j}^{i}$ as much as possible,
we compute these probabilities using the following procedure:
For all regions $i$ of PS that are covered only by the inclusive sample, and by none of the $\HT$- or $N_{\jet}$-samples,
we obtain the probability $P^{i}$ by determining the fraction of events in the inclusive sample that fall into PS region $i$.
For PS regions $i$ that are covered by the inclusive sample and by one or more $N_{\jet}$- or $\HT$-samples,
we determine the probabilities $P^{i}$ and $P_{j}^{i}$ by the method of least squares~\cite{Cowan:1998ji}.
When applying the least-squares method, we make the assumption that the following relation holds:
\begin{equation}
\sigma_{\incl} \times P^{i} = \sigma_{k} \times P_{k}^{i} \quad \forall k \,
\label{eq:lambda1}
\end{equation}
except for statistical fluctuations on the $P^{i}$ and $P_{k}^{i}$.
The $P^{i}$ and $P_{k}^{i}$ are obtained by determining the fraction of events in the inclusive and exclusive samples that fall into PS region $i$.
The symbol $k$ refers to those $N_{\jet}$- and $\HT$-samples that cover the PS region $i$
and the symbol $\sigma_{k}$ to the cross sections corresponding to these samples.
We denote the unknown true value of the left-hand-side (and equivalently of the right-hand-side) of Eq.~(\ref{eq:lambda1}) by the symbol $\lambda_{i}$
and use the symbols $r^{i}$ and $r_{k}^{i}$ to refer to the deviations (``residuals''), caused by statistical fluctuations,
between the true values of the probabilities $P^{i}$ and $P_{k}^{i}$ and the values obtained using the MC samples.
We further use the symbols $s^{i}$ and $s_{j}^{i}$ to denote the expected statistical fluctuations of these probabilities.
According to the least-squares method,
the best estimate for the value of $\lambda_{i}$ is obtained as solution to the equations:
\begin{eqnarray*}
\sigma_{\incl} \times \left( P^{i} + r^{i} \right) - \lambda_{i} & = & 0 \quad \mbox{and} \\
\sigma_{k} \times \left( P_{k}^{i} + r_{k}^{i} \right) - \lambda_{i} & = & 0 \quad \forall k \, ,
\end{eqnarray*}
subject to the condition that the sum of residuals
\begin{equation*}
S = \left( \frac{r^{i}}{s^{i}} \right)^{2} + \sum_{k} \left( \frac{r_{k}^{i}}{s^{i}} \right)^{2}
\end{equation*}
attains its minimal value.
The expected statistical fluctuations $s^{i}$ and $s_{j}^{i}$ of the probabilities $P^{i}$ and $P_{k}^{i}$
are given by the standard errors of the Binomial distribution~\cite{Cowan:1998ji}
\begin{equation*}
s^{i} = \sqrt{\frac{P^{i} \times (1 - P^{i})}{N_{\incl}}} \quad \mbox{and} \quad s_{k}^{i} = \sqrt{\frac{P_{k}^{i} \times (1 - P_{k}^{i})}{N_{k}}} \, .
\end{equation*}
The fluctuations decrease proportional to the inverse of the square-root of the number of events in the MC samples.
The solution for $\lambda_{i}$ is given by the expression:
\begin{equation}
\lambda_{i} = \frac{\alpha_{\incl}^{i} \times \sigma_{\incl} \times P^{i} + \sum_{k} \alpha_{k}^{i} \times \sigma_{k} \times P_{k}^{i}}{\alpha_{\incl}^{i} + \sum_{k} \alpha_{k}^{i}} \, ,
\label{eq:lambda2}
\end{equation}
from which the probabilities $P^{i} = \lambda_{i}/\sigma_{\incl}$ and $P_{k}^{i} = \lambda_{i}/\sigma_{k}$ follow.
The symbols $\alpha_{\incl}^{i}$ and $\alpha_{k}^{i}$ are defined as:
\begin{equation*}
\alpha_{\incl}^{i} = \frac{1}{\left( \sigma_{\incl} \times s^{i} \right)^{2}} \quad \mbox{and} \quad \alpha_{k}^{i} = \frac{1}{\left( \sigma_{k} \times s_{k}^{i} \right)^{2}}
\end{equation*}
and act as weights in the expression on the right-hand-side of Eq.~(\ref{eq:lambda2}),
which has the form of a weighted average.
We use the symbols $\alpha_{\incl}^{i}$ and $\alpha_{k}^{i}$ to refer to these weights,
in order to distinguish them from the stitching weights $w^{i}$ given by Eq.~(\ref{eq:weight_incl}).

The numerical values of the probabilities $P^{i}$ and of the stitching weights $w^{i}$ are given in Tables~\ref{tab:probabilities_WJets_vs_Njet_and_HT}
and~\ref{tab:weights_WJets_vs_Njet_and_HT}
% in the appendix
.

Distributions in $\pT$ of the leading and subleading jet,
in the multiplicity of jets, and in the observable $\HT$ 
for the sum of inclusive plus exclusive samples are compared to the distributions obtained for the inclusive sample in Fig.~\ref{fig:controlPlots_WJets_vs_Njet_and_HT}.
We use the same fill pattern for all $N_{\jet}$-samples, another pattern for all $\HT$-samples and a third pattern for the stitched inclusive sample,
so that one can better see where each of the stitched samples contribute the most.
In the lower part of each figure, we again show the difference between the background estimates obtained from the inclusive sample and obtained by using our stitching procedure,
and also the respective statistical uncertainties.
As one would expect, the addition of samples simulated in ranges in $\HT$ to the example given in Section~\ref{sec:WJets_vs_Njet}
reduce the statistical uncertainties in particular in the tail of the $\HT$ distribution.

Physics analyses that search for new particles of high mass, which decay to high $\pT$ jets, are thus best served by producing MC samples in ranges in $\HT$,
whereas the samples binned in $N_{\jet}$ are particularly well suited when the signal process under study features a large number of low $\pT$ jets.

\begin{table}[h!]
\begin{center}
\def\arraystretch{1.3}
\begin{tabular}{l|c|c|c}
\hline
\multirow{2}{20mm}{Sample} & Index & Number    & Cross                    \\
                           & $j$   & of events & section [nb]$^{\dagger}$ \\
\hline
\hline
$  70 < \HT <  100$~\GeV   &  $5$  &            $10^{6}$ & $1.50 \times 10^{3}$ \\
$ 100 < \HT <  200$~\GeV   &  $6$  &            $10^{6}$ & $1.62 \times 10^{3}$ \\
$ 200 < \HT <  400$~\GeV   &  $7$  & $5   \times 10^{5}$ & $479$ \\
$ 400 < \HT <  600$~\GeV   &  $8$  & $2.5 \times 10^{5}$ & $67.4$ \\
$ 600 < \HT <  800$~\GeV   &  $9$  &            $10^{5}$ & $15.1$ \\
$ 800 < \HT < 1200$~\GeV   & $10$  &            $10^{5}$ & $6.36$ \\
$1200 < \HT < 2500$~\GeV   & $11$  &            $10^{5}$ & $1.27$ \\
$       \HT > 2500$~\GeV   & $12$  &            $10^{5}$ & $9.41 \times 10^{-3}$ \\
\hline
\end{tabular}
\end{center}
$^{\dagger}$ Computed at LO accuracy in pQCD, then scaled to NNLO
\caption{
  Number of events in the $\PW \to \Plepton\Pnu$ samples simulated produced in ranges in $\HT$ and corresponding cross sections.
}
\label{tab:samples_WJets_vs_Njet_and_HT}
\end{table}

\begin{figure}
\setlength{\unitlength}{1mm}
\begin{center}
\begin{picture}(180,82)(0,0)
\includegraphics*[height=82mm]{plots/regions_WJets_vs_Njet_and_HT.pdf}
\end{picture}
\end{center}
\caption{
  Definition of the PS regions $i$ in the plane of $N_{\jet}$ versus $\HT$,
  %for the case that $\PW \to \Plepton\Pnu$ samples generated at LO accuracy in pQCD are stitched based on the observables $N_{\jet}$ and $\HT$.
  for the case of $\PW \to \Plepton\Pnu$ samples that are stitched based on the observables $N_{\jet}$ and $\HT$.
}
\label{fig:regions_WJets_vs_Njet_and_HT}
\end{figure}

\begin{figure}
\setlength{\unitlength}{1mm}
\begin{center}
\begin{picture}(180,182)(0,0)
\put(6.5, 100.0){\mbox{\includegraphics*[height=82mm]{plots/WJets_lead_stack_wRatio_log.pdf}}}
\put(81.5, 100.0){\mbox{\includegraphics*[height=82mm]{plots/WJets_sublead_stack_wRatio_log.pdf}}}
\put(6.5, 4.0){\mbox{\includegraphics*[height=82mm]{plots/WJets_njet_stack_wRatio_log.pdf}}}
\put(81.5, 4.0){\mbox{\includegraphics*[height=82mm]{plots/WJets_ht_stack_wRatio_log.pdf}}}
\put(43.0, 96.0){\small (a)}
\put(118.0, 96.0){\small (b)}
\put(43.0, 0.0){\small (c)}
\put(118.0, 0.0){\small (d)}
\end{picture}
\end{center}
\caption{
  Distributions in $\pT$ of the (a) leading and (b) subleading jet,
  in (c) the multiplicity of jets and in (d) the observable $\HT$,
  %for the case that $\PW \to \Plepton\Pnu$ samples generated at LO accuracy in pQCD are stitched based on the observables $N_{\jet}$ and $\HT$.
  for the case of $\PW \to \Plepton\Pnu$ samples that are stitched based on the observables $N_{\jet}$ and $\HT$.
  The event yields are computed for an integrated luminosity of $140\fbinv$.
}
\label{fig:controlPlots_WJets_vs_Njet_and_HT}
\end{figure}


