\def\verPreprint{1}
\def\verPAPER{2}
\def\ver{1}

\ifx\ver\verPreprint
\documentclass[a4paper,english,11pt]{article}
\usepackage[bindingoffset=0.5cm,left=2.5cm,right=2.5cm,top=2.5cm,bottom=2.5cm,footskip=1.0cm]{geometry}
\usepackage{lineno,amsmath,bm,multirow,amssymb,authblk,graphicx,newclude,xspace,hyperref,rotating}
\fi
\ifx\ver\verPAPER
\documentclass[1p]{elsarticle}
\usepackage{lineno,hyperref,amsmath,hepnames,bm,multirow,amssymb,xspace,rotating}
\fi

%% Prevent breaking of footnotes to pages that are different from the page where the footnote symbol appears
\interfootnotelinepenalty=10000

\modulolinenumbers[5]

%%\journal{Journal of \LaTeX\ Templates}

%%%%%%%%%%%%%%%%%%%%%%%
%% Elsevier bibliography styles
%%%%%%%%%%%%%%%%%%%%%%%
%% To change the style, put a % in front of the second line of the current style and
%% remove the % from the second line of the style you would like to use.
%%%%%%%%%%%%%%%%%%%%%%%

%% Numbered
%\bibliographystyle{model1-num-names}

%% Numbered without titles
%\bibliographystyle{model1a-num-names}

%% Harvard
%\bibliographystyle{model2-names.bst}\biboptions{authoryear}

%% Vancouver numbered
%\usepackage{numcompress}\bibliographystyle{model3-num-names}

%% Vancouver name/year
%\usepackage{numcompress}\bibliographystyle{model4-names}\biboptions{authoryear}

%% APA style
%\bibliographystyle{model5-names}\biboptions{authoryear}

%% AMA style
%\usepackage{numcompress}\bibliographystyle{model6-num-names}

%% `Elsevier LaTeX' style
\bibliographystyle{elsarticle-num}
%%%%%%%%%%%%%%%%%%%%%%%

%%%%%%%%%%%%%%%%%%%%%%%
%% Custom latex macros
%%%%%%%%%%%%%%%%%%%%%%%

\ifx\ver\verPreprint
\newcommand{\PW}{\ensuremath{\textrm{W}}\xspace}
\newcommand{\PZ}{\ensuremath{\textrm{Z}}\xspace}
\newcommand{\PGg}{\ensuremath{\gamma}\xspace}
\newcommand{\PHiggs}{\ensuremath{\textrm{H}}\xspace}
\newcommand{\Pp}{\ensuremath{\textrm{p}}\xspace}
\newcommand{\Plepton}{\ensuremath{\ell}\xspace}
\newcommand{\Pnu}{\ensuremath{\nu}\xspace}
\newcommand{\Pgt}{\ensuremath{\tau}\xspace}
\newcommand{\Pbottom}{\ensuremath{\textrm{b}}\xspace}
\newcommand{\APbottom}{\ensuremath{\bar{\textrm{b}}}\xspace}
\newcommand{\Ptop}{\ensuremath{\textrm{t}}\xspace}
\newcommand{\APtop}{\ensuremath{\bar{\textrm{t}}}\xspace}
\fi

\newcommand{\Pggx}{\ensuremath{\PGg^{*}}\xspace}
\newcommand{\PZggx}{\ensuremath{\PZ/\Pggx}\xspace}
\newcommand{\pT}{\ensuremath{p_{\textrm{T}}}\xspace}
\newcommand{\pThat}{\ensuremath{\hat{p}_{\textrm{T}}}\xspace}
%\newcommand{\kT}{\ensuremath{k_{\textrm{T}}}\xspace}
\newcommand{\kt}{\ensuremath{k_{\textrm{t}}}\xspace}
\newcommand{\ptmiss}{\ensuremath{p_{\textrm{T}}^\textrm{miss}}\xspace}
\newcommand{\HT}{\ensuremath{H_{\mathrm{T}}}\xspace}
\newcommand{\GeV}{\ensuremath{\textrm{GeV}}\xspace}
\newcommand{\TeV}{\ensuremath{\textrm{TeV}}\xspace}
\newcommand{\data}{\ensuremath{\textrm{data}}\xspace}
\newcommand{\mc}{\ensuremath{\textrm{mc}}\xspace}
\newcommand{\incl}{\ensuremath{\textrm{incl}}\xspace}
\newcommand{\jet}{\ensuremath{\textrm{jet}}\xspace}
\newcommand{\pileup}{\ensuremath{\textrm{PU}}\xspace}
\newcommand{\Poisson}{\ensuremath{\textrm{Poisson}}\xspace}
\newcommand{\Nbar}{\ensuremath{\bar{N}}\xspace}
\newcommand{\wbar}{\ensuremath{\bar{w}}\xspace}
\newcommand{\Born}{\ensuremath{\textrm{born}}\xspace}
\newcommand{\ME}{\ensuremath{\textrm{me}}\xspace}
\newcommand{\MGvATNLO}{\textsc{MadGraph5}\_aMC@NLO\xspace}
\newcommand{\PYTHIA}{\textsc{Pythia}\xspace}
\newcommand{\HERWIG}{\textsc{Herwig}\xspace}
\newcommand{\HERWIGpp}{\textsc{Herwig}++\xspace}
\newcommand{\TOPpp}{\textsc{Top}++\xspace}
\newcommand{\POWHEG}{\textsc{Powheg}\xspace}
\newcommand{\ALPGEN}{\textsc{Alpgen}\xspace}
\newcommand{\SHERPA}{\textsc{Sherpa}\xspace}
\newcommand{\FEWZ}{\textsc{FEWZ}\xspace}
\newcommand{\MCFM}{\textsc{MCFM}\xspace}
\newcommand{\cf}{cf.\xspace}
\newcommand{\ie}{i.e.\xspace}
\newcommand{\eg}{e.g.\xspace}
\newcommand{\fbinv}{\ensuremath{\textrm{~fb}^{-1}}\xspace}
\def\TReg{\textsuperscript{\textregistered}}
\usepackage{array}
\newcolumntype{C}[1]{>{\centering\arraybackslash}p{#1}}
%%%%%%%%%%%%%%%%%%%%%%%

\begin{document}

\ifx\ver\verPAPER
\begin{frontmatter}
\fi

\title{Stitching Monte Carlo samples}

%% Group authors per affiliation:

\ifx\ver\verPreprint
\author[1]{Karl Ehat\"aht}
\author[1]{Christian Veelken}
\affil[1]{National Institute for Chemical Physics and Biophysics, 10143 Tallinn, Estonia}
\fi
\ifx\ver\verPAPER
\author[tallinn]{Karl Ehat\"aht}
\ead{karl.ehataht@cern.ch}
\author[tallinn]{Christian Veelken}
\ead{christian.veelken@cern.ch}
\address[tallinn]{National Institute for Chemical Physics and Biophysics, 10143 Tallinn, Estonia}
\fi

\ifx\ver\verPreprint
\maketitle
\fi

\begin{abstract}
Monte Carlo (MC) simulations are extensively used for various purposes in modern high-energy physics (HEP) experiments.
Precision measurements of established Standard Model processes or searches for new physics often require the collection of vast amounts of data.
It is often difficult to produce MC samples containing an adequate number of events to allow for a meaningful comparison with the data, 
as substantial computing resources are required to produce and store such samples.
One solution often employed when producing MC samples for HEP experiments 
is to partition the phase space of particle interactions into multiple regions 
and produce the MC samples separately for each region.
This approach allows to adapt the size of the MC samples to the needs of physics analyses that are performed in these regions.
In this paper we present a procedure for combining MC samples that overlap in phase space.
The procedure is based on applying suitably chosen weights to the simulated events.
We refer to the procedure as ``stitching''.
The paper includes different examples for applying the procedure to simulated proton-proton collisions at the CERN Large Hadron Collider.
\end{abstract}

\ifx\ver\verPAPER
\end{frontmatter}
\fi

\clearpage

\linenumbers

%\begingroup
%\let\clearpage\relax
\include*{introduction}
\include*{stitching_weights}
\include*{examples}
\include*{examples_lo}
\include*{examples_trigger}
\include*{summary}
\include*{acknowledgements}

\appendix
\include*{appendix}

\clearpage

\bibliography{mcStitching}
%\endgroup

\end{document}
